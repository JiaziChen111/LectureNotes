%\subsection{Discrete games}
%
%Let $\pi(\cdot)$ be the profit function (continuation value) earned in the second period, $d_i$ and $d_{-i}$ be agent $i$'s and its competitors' discrete decisions respectively, $y_i$ be the set of variables that affect the agent's profits, $D_i$ be the choice set (of decisions) and $I_i$ the information set. Further, denote as $\mathcal{E}\left[ \cdot | I_i \right] $ the agent's expectation (note that we do not use the usual expectation term because we want to differentiate from the ``econometric'' expectation we usually use).
%
%In order to estimate this discrete game, we need two conditions to hold:
%
%\subsubsection{Nash condition}
%
%The Nash condition states that: $$ \sup_{d\in\D_i, d\neq d_i} \mathcal{E}\left[ \pi(d, d_{-i}, y_i, \theta) | I_i \right] \leq \mathcal{E}\left[ \pi(d_i, d_{-i}, y_i, \theta) | I_i \right]  \text{ for all } i = 1, ..., n $$ In words, this condition ensures that the observed decision $d_i$ was at least among the best (in expectations) compared to alternatives, given the information set available.
%
%Note that this condition does not restrict the choice set to be discrete (and thus could apply to more settings than entry). Moreover, while this condition imply some kind of rationality, it does not imply anything about uniqueness of the solution (i.e. the observed decision might be one of multiple equilibria).
%
%\subsubsection{Counterfactual condition}
%
%The counterfactual condition allows us to recover what would have happened in the case of other decisions: $$ d_{-i} = d(d_i, z_i) \text{ and } y_i = y(z_i, d) $$ where $z_i$ is exogenous of $d_i$. This condition implies that conditional on the information set, beliefs about the competitors' actions depend only on the agent's decision and exogenous variables (that do not change with the decision).
%
%In the case of simultaneous games, notice that $d(\cdot)$ is just the observed action $d_{-i}$ for any $d_i$.
%
%\subsubsection{Implications}
%
%Let $d' \in D_i$ be any alternative choice and let $$ \Delta\pi(d_i, d', d_{-i}, z_i) \equiv \pi(d_i, d_{-i}, z_i) - \pi(d', d_{-i}, z_i) $$ then, using the Nash and the counterfactual conditions, we have that: $$\mathcal{E}\left[ \Delta\pi(d_i, d', d_{-i}, y_i) | I_i \right] \geq 0 \text{ for all } d' \in D_i $$ While this implication seems straightforward considering the two conditions presented earlier, it will serve as a basis for the estimation algorithm. However, for that relation to be useful, we need to specify two more elements: (1) the relation between agents' expectations ($\mathcal{E}$) and observed sample moments ($\text{E}$) and (2) the functional form of $\pi$ in relation to the variables $z_i, d_i$ and $d_{-i}$ and how they can be described by observed variables.
%
%\subsection{Entry model with structural error}


