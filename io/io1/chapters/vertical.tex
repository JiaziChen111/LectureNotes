\section{Introduction}

Manufacturers rarely supply final consumers directly but are usually vertically separated by one or more intermediaries. In this particular type of setting, we often refer to the manufacturer as the upstream firm, and the intermediary as the downstream firm. This relationship is the same as in a typical market, considering the downstream firm as the customer and the upstream firm as the producer, thus leading to the same topics as usual: endogenous pricing, price discrimination, etc.

The main addition to the usual models is that now the downstream firm is not the end consumer, but rather an agent that will serve the end consumers, thus doing its own share of pricing, advertising, etc. Because these activities will affect the consumers, the upstream firm has incentives to control the downstream firms in some way, we call this ``vertical control''. There are several types of vertical restraints used by firm:\begin{itemize}
\item Exclusive territories: a single retailer is assigned to a ``territory'' (geographical or not) and has monopoly rights over the area.
\item Exclusive dealings: a retailer that chooses the upstream firm cannot sell nor carry any of the competitors' goods.
\item Full-line forcing: a dealer is committed to sell the whole product line of the upstream manufacturer.
\item Resale Price Maintenance (RPM): a dealer commits to retail prices (or a range) that will hold for the product. Equivalently, quantity forcing or rationing will commit the retailer on the quantity side.
\item Contractual arrangements: more flexible agreement between upstream and downstream firms to transfer the product. Profit and revenue sharing are the most common.
\end{itemize}

\section{Theoretical insights}

\subsection{Basic Framework}

Start with a simple model with a homogeneous good with demand given by $p = a - Q$. Moreover, both the upstream and downstream firms are monopolists. The downstream firm has a distribution cost equal to $d$ (the price it pays for the upstream good), while the upstream firm has a marginal cost equal to $c$.

\subsection{Externalities}

Because the downstream firm has a monopoly over retailing, its optimal strategy is to charge the monopoly price for the product. The manufacturer also has a monopoly over the production of the good, thus will charge a monopoly price to the retailer! 

From this example we can draw multiple results: \begin{itemize}
\item The upstream firm earns higher profits than the retailer.
\item The upstream firm would earn even more by selling directly to the market.
\item Total industry profits are lower than vertically integrated profits.
\end{itemize}
These results come from the presence of two markups! This is what we call double marginalization. Going around this issue can be done by including additional terms in the contracts (RPM, quantity forcing, etc.).



\subsection{Downstream Moral Hazard}



\subsection{Interbrand Competition and Legal Issues}



\section{Mortimer (2008)}



\section{Conlon and Mortimer (2017)}



\section{Ho, Ho and Mortimer (2012)}



\section{Crawford et al. (2017)}

