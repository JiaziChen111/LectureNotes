\section{Aguirregabiria (1999)}

\subsection{Summary}

\subsubsection{Background}

There is substantial evidence of price dispersion and staggering (prices are not flexible), most commonly explained by menu costs (changing prices is costly) and not perfect correlation in demand (different demand shocks across firms). Confirmed by theory of (S, s) rule (= target price and adjustment bands with adjustment only outside the bands). But what to make of evidence that retail firms might not use (S, s) rules, or that prices go down sometimes (with inflation, prices should always go up)? Most explanations rely on ``consumer-side'' phenomenons, not on inventories (supply-side)!

Maybe a model of (S, s) rule with inventories would explain some variation!

\subsubsection{Model}

Firms maximize profits as a function of expected sales, order costs, inventory costs and menu costs (all three are lump-sum!) = (S, s) rule is still optimal!

Without menu costs, inventories are perfectly negatively correlated with markups: giving evidence of price markdowns when inventory is high (right after an order) but no price staggering.

With menu costs, you get price staggering as well as markdowns.

\subsubsection{Data}

Monthly information on prices, sales, orders and inventories for every item = balanced panel data. Price data is monthly averages (for regular price and sales price).

\subsubsection{Empirical evidence}

Naive analysis and reduced form regressions show infrequent price changes, prevalence of sales promotions, infrequent orders, negative correlations between inventories and prices.

Using a Hotz-Miller approach, authors find the same results: data is consistent with high lump-sum cost of ordering and menu costs.

\subsubsection{Counterfactuals}

Removing these lump sum costs actually leads to more variability = they explain a lot of price staggering and price reductions! Demand might explain the rest.

\subsection{Discussion}



\section{Hendel and Nevo (2002)}

\subsection{Summary}

\subsubsection{Background}

When goods are storable, price variation might create bias in demand estimation. In fact, if prices decrease (during sales for example), then consumers might stockpile: creates large demand increase in the short run which, measured in long-run (when prices have gone back up) will inflate own-price elasticities. Cross-price elasticities be ambiguous. Because these elasticities are so important in industry analysis, you have to get them right!

This is the analog analysis to the Aguirregabiria (1999) paper where he studies price variation and inventories (= the supply side).

\subsubsection{Data}

Scanner data on store-level and household-level consumption in the detergent market. Hendel and Nevo observe price, quantities, some measure of advertising, sales, inventories (both at the store and household levels).

Reduced form analysis shows that duration between purchases has a positive effect on quantity purchased (= household do hold inventories); storage costs are negatively correlated with buying on sale (= not being able to store leads to less storage?); households buy more on sale, even when they do hold inventories already (= stockpiling).

\subsubsection{Model}

Do not model quantity purchases but only sizes! Dynamic discrete choice model with utility of purchase and inventory costs.