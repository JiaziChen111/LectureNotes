\section{Introduction}

Search is an alternative explanation to observing different market shares, enormous marketing budgets, etc. than simple product differentiation.

Search is defined as looking for another price ``quote'' for a homogenous product. Firms simultaneously choose prices, then consumers search among a number of prices to find the best one and purchase.

If search is sequential, then consumers would search until finding a price less than or equal to a reservation price $r$ such that the expected benefit of finding $r$ is equal to the search cost.

Diamond (1971) shows that this model is very interesting in that it leads to different results than typical Bertrand competition when $s>0$, the search cost is greater than 0. However, it does not converge to a Bertrand model as $s\to 0$.

\section{Other models}

\subsection{Diamond (1971)}

\begin{itemize}
\item Main result: unique NE is monopoly prices and no search!
\begin{itemize}
\item Independent of the number of firm and search cost (provided $s>0$).
\item Depends on inability to advertise price cuts.
\end{itemize}
\end{itemize}

\subsection{Stahl (1996)}

\begin{itemize}
\item Add a portion ($\mu$) of consumers as shopping-lovers $\Rightarrow$ price dispersion!
\begin{itemize}
\item No pure-strategy NE!
\item But very challenging to compute...
\end{itemize}
\end{itemize}

\subsection{Simpler model}

\begin{itemize}
\item Simplification of Stahl by adding: inelastic demand + fixed search costs.
\begin{itemize}
\item The upper bound of prices ensures that only shoppers search.
\item The lower bound of prices is marginal cost + a markup that depends negatively on the proportion of shoppers and positively the size of the search cost.
\end{itemize}
\end{itemize}

\section{Salz (2017)}

\subsection{Summary}

\subsubsection{Background}

Search costs in decentralized markets are important and incentivize the role for intermediaries to enter the market. But how do they influence market outcomes? Empirical study of the NYC trade waste industry. Companies generate waste and have to comply by law to finding a waste management service. Brokers might help with this process by awarding contracts on a competitive bidding process. Given that search costs are about 11-35\% of expenses, what is the effect of brokers on the rents in that market?

\subsubsection{Data}

Observation unit is a contract between company and waste management (from regulatory institution). Variables observed include zip code, negotiated price, quantity of waste and whether or not it was brokered. Some markets are very concentrated while others are very competitive.

There is a lot of price dispersion (even with the same observables), implying search intermediation is useful.

\subsubsection{Model}

Sequential game between customers and carters, where brokers are non-strategic. Customers observe search costs, carters observe their search cost (this is private information). The game plays as follows:\begin{enumerate}
\item Customers learn his search cost and carters learn their costs.
\item Customers decide between delegating search or performing a number of search (not sequential).
\item Carters submit prices to either broker (given a first-price auction) or to customers directly.
\end{enumerate}

\subsubsection{Assumptions}

\begin{itemize}
\item Simultaneous search rather than sequential.
\item No broker competition.
\item Contract length is fixed to two years: underestimation of search costs.
\item Number of bidders is fixed to observed winners: underestimate number of bidders $\Rightarrow$ search costs are underestimated.
\item Carters are grouped into types.
\end{itemize}

\subsubsection{Results}

Intermediaries create benefits to both customers who use brokers and those who do not. 

\subsection{Discussion}

\begin{itemize}
\item \textbf{Research questions:} (1) How do intermediaries affect buyers and sellers of trash management services? (2) What is the welfare effect?
\item \textbf{Goals of the paper:} develop a method (introduce empirical study of search market with intermediaries) and measure an effect (presence of intermediaries)
\item \textbf{Importance of the paper:} 1st empirical framework to study search including (1) intermediaries and (2) heterogeneous costs.
\item \textbf{Methodology:} Using FPA data (on brokers), able to separate production costs from search costs! (thus get heterogeneous costs).
\item \textbf{Theoretical foundations:}  simultaneous search model (fixed number of searches) to identify search costs and first-price auction in brokers market to identify carter costs. \begin{itemize}
\item Strengths: fits well with the data
\item Shortcomings: no long justification on these choices; no broker competition; no switching dynamics.
\end{itemize}
\item \textbf{Empirical strategy:} use a structural model to (1) instrument for the presence of brokers (no reduced form possible) and (2) measure welfare (by measuring costs).\begin{itemize}
\item Strengths: 
\item Shortcomings: (1) homogeneous brokers; (2) broker-customer relations are not observed (price is ``inferred'')
\end{itemize}
\item \textbf{Data:} Contract-level observations (ZIP code, price, quantity, date, brokerage, etc.); NO BROKER FEE!
\begin{itemize}
\item Exogenous: ZIP code, market size, demographics, quantity (!), market structure (types and costs of sellers), search costs, broker sets, fees, search sets (up to $M$).
\item Endogenous: number of searches, seller prices, contracts.
\end{itemize}
\item \textbf{Results:} Costs are lower for bigger customers (economies of scale); number of searches does not depend on quantity = returns to search are stable = cost is increasing?; brokers increase welfare. 
\end{itemize}