\section{Effects}

\subsection{Model 1: Investing and Harvesting}

\begin{itemize}
\item Firms ``invest'' in consumers by lowering their prices below marginal cost, then ``harvest'' their loyal base by extracting all surplus!
\begin{itemize}
\item Only if switching cost is high enough and perfect competition!
\item Welfare is intact (average price is still marginal cost)
\end{itemize}
\item No switching in equilibrium.
\end{itemize}

\subsubsection{Model 1a: Add heterogenous values}

\begin{itemize}
\item Now there is a DWL: consumers with high value of the good lose!
\item No switching in equilibrium.
\end{itemize}

\subsubsection{Model 1b: Add heterogenous switching costs}

\begin{itemize}
\item Identical welfare outcomes but more switching (those who have low draws of switching costs).
\end{itemize}

\subsection{Model 2: Hotelling Duopoly}

\begin{itemize}
\item Distance between consumer taste and firm is key to compute demand.
\begin{itemize}
\item Consumer has $\theta \in [0, 1]$ in both periods, firms are at $0$ and $1$.
\item Budget is $B$.
\item Transportation cost is $\tau$.
\end{itemize}
\end{itemize}

\subsubsection{Model 2a: Taste is known and stable}

\begin{itemize}
\item As if no switching costs (same equilibrium as pure Hotelling).
\item No switching in equilibrium.
\end{itemize}

\subsubsection{Model 2b: Taste in second period is unknown}

\begin{itemize}
\item Average price is half of previous model!
\begin{itemize}
\item This is because when the decision is made, products are not differentiated (taste unknown), thus firms need to ``invest'' harder.
\end{itemize}
\end{itemize}

\subsubsection{Model 2c: Consumer myopia}

\begin{itemize}
\item Same as model 2b!
\begin{itemize}
\item Same intuition, myopia means that second period does not come into decision, thus attracting consumers is harder.
\end{itemize}
\end{itemize}

\section{Measurement}

\subsection{Measuring switching costs or inertia}

\begin{itemize}
\item Challenge: is it switching costs or persistent heterogenous preferences?
\begin{itemize}
\item Osborne (2011): look at previous period event's effect on current period (price cut increases probability to buy again? = switching costs).
\item Handel (2013): look at difference between old and new consumers (if old buy more than new = inertia)
\end{itemize}
\end{itemize}

\subsection{Decomposing inertia}

\begin{itemize}
\item Inertia is: switching costs, habit formation, search costs, learning, inattention, etc.
\begin{itemize}
\item Usually only one form is considered (others assumed away) = bad (Wilson, 2012)
\end{itemize}
\end{itemize}

\subsection{Comments on Wilson (2012)}

\begin{itemize}
\item Insight: a 1\$ search cost has more impact than a 1\$ switching cost because:
\begin{itemize}
\item Searching will cause the cost with probability 1, while switching is only incurred if better option available
\item Potential of multiple searches, while only one switch
\item Inversting from firms might help consumers offset cost
\end{itemize}
\item Thus, design a ``quick and easy'' method for estimating search and switch costs:
\end{itemize}

\section{Honka (2014)}

\subsection{Summary}

\subsubsection{Background}

Insurance markets can be inefficient for many reasons, two of them being search costs and switching costs (market frictions). What is their value in the auto insurance market? And do they affect consumer choice and welfare? are the two questions Honka (2014) answers. Auto insurance market is perfect in the sense that it includes both (very high retention rate). Honka puts both types of costs in her model: new thing in the literature!!

\subsubsection{Data}

the observation unit is a transaction, with contract observables (prices, premium, etc.), consumer observables (previous contract, number of quotes, by whom, and demographics, etc.). Consideration set is very important to identify search costs, previous contract to identify switching costs (but no panel data).

\subsubsection{Model}

Multinomial logit model with a search + previous insurer consideration set. There a ``first-step'' search model based on expected utility to decide how many searches.

\subsubsection{Estimation}

Simulated semi-parametric way of recovering number of searches (analogous to ordered probit?). With consumer beliefs, recover search costs. Finally, switching costs are inside RUM model.

\subsubsection{Assumptions}

Main model assumptions are: (1) search and purchase are conditional on coverage (no search across coverages); (2) search is used to discover prices only and (3) search is simultaneous rather than sequential.

Other assumptions include: static preferences (utility fixed effect is not correlated with previous choice); switching costs are not exactly identified (confounded with heterogenous persistent preferences.

\subsubsection{Results}

Search costs are three times higher than switching if made in person, but comparable if made through internet. Search costs are what drives most of the retention. Both have negative effects on welfare.