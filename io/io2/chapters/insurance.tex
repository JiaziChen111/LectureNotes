\section{Cohen and Einav (2007)}

\subsection{Summary}

\subsubsection{Background}

There is little empirical research on the link between risk aversion and individual characteristics (how different agents might have different aversion to risk). But it is possible to recover these by looking at market outcomes in which participants had to make a choice between risky outcomes.

\subsubsection{Data}

Auto insurance company in Israel, data for 100k new customers, only focus on the first choice (to remove discussion about switching and search costs). Observation unit is thus transaction-level with all customer info that insurance provider got + menu of choice (four policies and premia), but also the length of contract and the actual use of the contract (claims and amounts claimed).

\subsubsection{Model}

Using Expected Utility Theory to derive utility function (vNM style). Two main factors to the choice: risk aversion and claim risk (tradeoff is given by an equation recoverable from the data). Estimation by ML.

\subsubsection{Assumptions}

Claims are drawn from a Poisson distribution estimated on consumer characteristics. No moral hazard.

\subsubsection{Results}

Risk averse is heterogenous (dependent on observables). Claim risk is heterogenous.

\subsection{Discussion}

\begin{itemize}
\item Authors add dimensions to the question of adverse selection (how are people risk averse, are they different, etc.)
\end{itemize}

\section{Einav, Finkelstein and Cullen (2010)}

\subsection{Summary}

\subsubsection{Background}

Welfare loss in insurance market is not to be proved anymore, but it is still difficult to quantify (because of hidden information mostly). With fewer assumptions, yet enough structure, this paper provides a methodology to evaluate welfare in insurance market.

\subsubsection{Model}

Given two choices of policies, normalize price and quantities to represent the difference between the two contracts. Then, crossing between demand and MC is the efficient, while crossing with AC is equilibrium. The difference is the deadweight loss.

\subsubsection{Data}

Employer-provided health insurance at Alcoa.

\subsubsection{Results}

Risk averse is heterogenous (dependent on observables). Claim risk is heterogenous.