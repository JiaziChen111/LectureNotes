\section{Cohen and Einav (2007)}

\subsection{Summary}

\subsubsection{Background}

There is little empirical research on the link between risk aversion and individual characteristics (how different agents might have different aversion to risk). But it is possible to recover these by looking at market outcomes in which participants had to make a choice between risky outcomes.

\subsubsection{Data}

Auto insurance company in Israel, data for 100k new customers, only focus on the first choice (to remove discussion about switching and search costs). Observation unit is thus transaction-level with all customer info that insurance provider got + menu of choice (four policies and premia), but also the length of contract and the actual use of the contract (claims and amounts claimed).

\subsubsection{Model}

Using Expected Utility Theory to derive utility function (vNM style). Two main factors to the choice: risk aversion and claim risk (tradeoff is given by an equation recoverable from the data). Estimation by ML.

\subsubsection{Assumptions}

Claims are drawn from a Poisson distribution estimated on consumer characteristics. No moral hazard.

\subsubsection{Results}

Risk averse is heterogenous (dependent on observables). Claim risk is heterogenous.

\subsection{Discussion}

\begin{itemize}
\item \textbf{Research questions:} (1) How risk-averse are individuals? (2) How do risk preferences vary across individuals (observed and unobserved)?
\item \textbf{Goals of the paper:} develop a method (introduce structural estimation of demand for risk) and measure an effect (adverse selection)
\item \textbf{Importance of the paper:} Adverse selection and heterogeneity of preferences towards risk are very important, 1st paper to measure empirically how important they are.
\item \textbf{Theoretical foundations:} Expected utility theory. \begin{itemize}
\item Strengths: 
\item Shortcomings: does it really work? (Uzi's class in micro theory). Where is moral hazard?
\end{itemize}
\item \textbf{Empirical strategy:} use a structural model to recover unobserved variables (risk using outcome data and risk preferences using contract choice data). \begin{itemize}
\item Strengths: 
\item Shortcomings:
\end{itemize}
\item \textbf{Data:} From an insurance firm, contract-level data (driver characteristics, contract choice set, chosen contract, claims)
\begin{itemize}
\item Exogenous: claims (no moral hazard), price, customers.
\item Endogenous: contract choice.
\end{itemize}
\item \textbf{Results:} Median driver is risk-neutral, mean driver is risk-averse. Risk aversion is correlated to risk (= adverse selection). More heterogeneity in risk aversion than in risk. 
\end{itemize}

\section{Einav, Finkelstein and Cullen (2010)}

\subsection{Summary}

\subsubsection{Background}

Welfare loss in insurance market is not to be proved anymore, but it is still difficult to quantify (because of hidden information mostly). With fewer assumptions, yet enough structure, this paper provides a methodology to evaluate welfare in insurance market.

\subsubsection{Model}

Given two choices of policies, normalize price and quantities to represent the difference between the two contracts. Then, crossing between demand and MC is the efficient, while crossing with AC is equilibrium. The difference is the deadweight loss.

\subsubsection{Data}

Employer-provided health insurance at Alcoa.