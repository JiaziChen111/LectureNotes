\section{Dana (1999)}

\subsection{Model}

\begin{itemize}
\item Consumers on a continuum of types with unit demand.
\item Firm has marginal production and capacity cost.
\item Firm chooses price before learning demand.
\item Proportional rationing imply all types have access and buy to the good proportionately.
\end{itemize}

\subsection{Residual Demand}

\begin{itemize}
\item Residual demand is very important:
\begin{itemize}
\item Start with any price, say $\tilde p$: some people buy, some don't, say $\tilde q$ have bought.
\item Residual demand at another price, say $p$ is not simply base demand - $\tilde q$, because some of those who bought at $\tilde p$ would not have bought at $p$ = use proportional rationing to determine residual demand
\end{itemize}
\end{itemize}

\subsection{Perfect Competition}

\begin{itemize}
\item Competitive market = profit is 0 (but probability!)
\item Price dispersion: from $\underline{p}$ to $\bar p$
\end{itemize}

\subsection{Monopoly}

\begin{itemize}
\item Market power imply markup: prices support is narrower!
\end{itemize}

\subsection{Price Dispersion and Market Structure}

\begin{itemize}
\item Two results:
\begin{itemize}
\item Support of prices widens with competition.
\item Variance of prices increases with competition (given linear demand).
\end{itemize}
\item In summary: price dispersion increases with PTR (which increases with competition).
\end{itemize}

\section{Leslie (2004)}

\subsection{Background}

The paper answers the question of welfare effects of price discrimination (consumer = ambiguous; firms = positive). Broadway play where second and third degree price discrimination. Second degree happens because of different seats are offered at different prices based on quality (nonlinear pricing as in Dana (1999)). Third degree is targeted coupons. Finally, discount sales for day-of-performance tickets is damaged goods. Same marginal cost for all seats but capacity costs (again, as in the Dana paper).

\subsection{Data}

Unit of observation is a seat (price, quality, discount, etc). Aggregate discount into two categories (coupons or booth). Aggregate advertising is observed. Competing plays attendance is observed. Finally, consumer variables in the NYC region are observed.

\subsection{Model}

Product space approach.

\subsection{Assumptions}



\subsection{Results}

Main result is that price discrimination improves welfare! But all types of PD are suboptimally designed.

\subsection{Discussion}

\begin{itemize}
\item \textbf{Research questions:} What are the welfare consequences of price discrimination for Seven Guitars on Broadway?
\item \textbf{Goals of the paper:} measure an effect (price discrimination) and maybe slightly interested in policy implications.
\item \textbf{Importance of the paper:} empirical framework oriented towards measure of welfare effects of price discrimination.
\item \textbf{Theoretical foundations:} price discrimination models with 3rd degree (market segmentation) and 2nd degree (self-selection/nonlinear pricing) price discrimination  \begin{itemize}
\item Strengths: (1) fits market details; (2) adds outside option to theoretical model (= oligopoly?); (3) no selling out means increasing welfare through sales is possible (Tirole on welfare).
\item Shortcomings: (1) Dana (1999) with demand uncertainty is missing (perfect framework); (2) no dynamics in consumer demand (perishable good); (3) relies on functional form assumptions.
\end{itemize}
\item \textbf{Empirical strategy:} use a structural model to study counterfactuals (measure welfare).\begin{itemize}
\item Strengths: No censoring in data (did not sell out so much)
\item Shortcomings: 
\end{itemize}
\item \textbf{Data:} (1) Show-level (tickets sold, prices, ads, quality, etc.) (2) Market-level (other shows' sales, customer incomes, demographics, etc.)
\begin{itemize}
\item Exogenous: ads, quality, other shows' sales, consumer variables
\item Endogenous: prices, sales
\end{itemize}
\item \textbf{Results:} Price discrimination has a positive effect on welfare (in general), mostly through profits. All PD schemes are ``suboptimally'' designed.
\end{itemize}