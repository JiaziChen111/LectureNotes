\section{Two types}

This section covers the model of nonlinear pricing where a monopolist offers a menu of two quantity/quality-price pairs to consumers of two types ($H$, with high value for the good and $L$ with low value).

Without loss of generality, we consider contracts such that each type chooses a contract and has no incentives to deviate. The monopolist chooses a quantity/quality and a price for each contracts, such that its profits are maximized. We get a problem with three elements: (1) an optimization problem, (2) a set of participation constraint (making sure each type buys the contract) and (3) an incentive constraint (making sure no type deviates).

The results of this model tells us that the optimal contracts are designed in such a way that the quantity/quality of the highest type is not distorted, while for the lowest type, they will receive a lower quantity/quality than their first best!

\section{Continuous types}

The continuous types model has the same structure as the one presented in the previous section, however, now types lie on a continuum from lowest to highest. As before, the model is separated in the three same parts and display a similar distrortion where consumers of the highest types get their first best contract, while the lowest types get lower quantity/quality.

\section{Crawford and Shum (2007)}

\subsection{Summary}

The research question of this paper is: ``To what extent is quality degradation prevalent in cable TV markets? And what are the effects of regulation on this issue?''

They add quality to the decision of the monopolist (add a dimension). Because of imperfect competition, we might think that quality will also be distorted (as prices are), which would create welfare losses. This is the framework of Mussa-Rosen, which is applied to the setting of cable TV.

\subsubsection{Mussa-Rosen}

There are three types of consumers (one to allow some consumers to not care about cable TV), and two contracts. The model displays distortion for the lowest types and no distortion at the top. They further go to show that this result holds even if consumer types are continuous if qualities are discrete.

Regulation is set up as a constraint on the optimization problem such that quality cannot go lower than a certain point. This turns out to only restrict the lowest quantity, while the top quantity stays undistorted.

\subsubsection{Cable TV industry}

Contracts are based on bundles of networks. Basic service is the one that everyone has, then you can buy extended service or premium (maps well to theory presented before, but premium is ignored because horizontal differentiation).

Bundle quality is measured in two ways: number of networks in said bundle (assuming same underlying quality) or through the implied values from consumer distribution (a first-stage problem from Mussa-Rosen).

\subsubsection{Empirical model}



\subsubsection{Results}

Quality distortion is present. Regulation mitigates the problem.

\subsection{Discussion}

\begin{itemize}
\item \textbf{Research questions:} (1) how much do cable TV distributors ``degrade'' quality? (2) how do local regulations affect quality distortion?
\item \textbf{Goals of the paper:} measure an effect (incentives to distort quality) and answer a policy question (quality regulation)
\item \textbf{Importance of the paper:} 1st empirical framework oriented towards measure of quality degradation.
\item \textbf{Theoretical foundations:} mainly Mussa and Rosen (1978), which is a model of monopoly facing discrete types of consumers, choosing prices and quantity/quality \begin{itemize}
\item Strengths: monopoly setting fits well with institutional details.
\item Shortcomings: (1) lack of horizontal preferences (2) truncation of ``real'' high types (3) some bundles do not fit model.
\end{itemize}
\item \textbf{Empirical strategy:} use a structural model to (1) infer quality (unobservables) and (2) study counterfactuals.\begin{itemize}
\item Strengths: 
\item Shortcomings: (1) lack of horizontal preferences (2) truncation of ``real'' high types (3) functional form assumptions drive the results and (4) reduced-form could be used more.
\end{itemize}
\item \textbf{Data:} Cable system-market level (prices, shares, content, market size, demographics)
\begin{itemize}
\item Exogenous: market size, demographics.
\item Endogenous: prices, shares, content (quality).
\end{itemize}
\item \textbf{Results:} Degradation exists, gets better with stricter regulation.
\end{itemize}

