\section{Introduction, model and assumptions}

\begin{definition}[Data sample]
A data sample is a set of $n$ observations, denoted by the subscript $i$. Its elements are random variables $(Y_i, X_i)$ where $X_i$ can be a vector.
\end{definition}

In econometrics, we commonly view these observations as iid draws from a common distribution called the Data-generating process (or DGP). 

Consider the a model of a variable $Y$ explained by $k$ regressors with $n$ observations: $$Y_i = \beta_1 + \beta_{2}X_{2i} + ... + \beta_kX_{ki} + e_i$$ which can be written in its matrix form:\[
\begin{bmatrix}
    Y_{1}\\
    Y_{2}\\
    \vdots\\
    Y_{n}
\end{bmatrix}
=
\begin{bmatrix}
    1       & X_{21} & X_{31} & \dots & X_{k1} \\
    1       & X_{22} & X_{32} & \dots & X_{k2} \\
    \vdots & \vdots & \vdots & \ddots & \vdots \\
    1       & X_{2n} & X_{3n} & \dots & X_{kn}
\end{bmatrix}
\cdot
\begin{bmatrix}
    \beta_1 \\
    \vdots \\
    \beta_k
\end{bmatrix}
+ 
\begin{bmatrix}
    e_{1}\\
    e_{2}\\
    \vdots\\
    e_{n}
\end{bmatrix}
\]
\begin{definition}[OLS estimator]
The OLS estimator $\hat\beta$ of the true parameter $\beta$ is the vector that minimizes the sum of squared residuals:\begin{align*}
\hat\beta & \in \operatorname{arg}\min_{\beta} e'e\\
& \in \operatorname{arg}\min_{\beta} (Y -  X\beta)'(Y -  X\beta)
\end{align*} 
\end{definition}

\begin{remark}
If the matrix given by $X'X$ is invertible, then the value of the OLS estimator $\hat\beta$ is:$$\hat\beta=(X'X)^{-1}X'Y$$
\end{remark}
\begin{proof}
In order to solve the model, we can simplify the objective function to $Y'Y - Y'\beta X - \beta'X'Y + \beta'X'X\beta $ where each term is a $1\times 1$ matrix. The FOC gives:\begin{align*}
\frac{\partial }{\partial \beta} = 0 & \Leftrightarrow -2X'Y + 2X'X\hat\beta = 0 \\ & \Leftrightarrow X'Y=X'X\hat\beta \\ & \Leftrightarrow \hat\beta=(X'X)^{-1}X'Y
\end{align*}
\end{proof}

\begin{remark}
In the particular case of $k=2$ so that $y = a + bx +e$. We have that: \begin{align*}
\hat b & = \frac{\sum_{i=1}^{n}(x_i - \bar x)(y_i - \bar y)}{\sum_{i=1}^{n}(x_i - \bar x)^2}  = \frac{\cov{x,y}}{\V{x}}\\
\hat a & = \bar y - \hat b \bar x
\end{align*}
\end{remark}

\begin{proof}

\end{proof}

\begin{definition}[Fitted values and residuals]
The fitted value $\hat{Y}$ is defined by:$$\hat{Y} = X\hat{\beta}$$ This variable is not a predictor of $Y$ since it is a function of the sample only. The residuals are $\hat{e} = Y - \hat{Y} = Y - X\hat{\beta}$. They are different from errors $e$ which are unobservable parameters of the regression.
\end{definition}

\begin{remark}
In a linear regression context, we have: $$X'\hat e = 0$$ This implies that if $X$ contains a column of ones (i.e. there is a constant in the model), then $\avg{1}{n}\hat e_i = 0$.
\end{remark}

\begin{proof}
We can easily see that:\begin{align*}
X'\hat e & = X'(Y - X\hat\beta) = X'Y - X'X\hat\beta = X'Y - X'X(X'X)^{-1}X'Y = 0 \\ & \Leftrightarrow
\begin{bmatrix}
    1 & 1 & \dots & 1 \\
    X_{21} & X_{22} & \dots & X_{2n} \\
    \vdots & \vdots & \ddots & \vdots \\
   X_{k1} & X_{k2} & \dots & X_{kn}
\end{bmatrix}
\cdot 
\begin{bmatrix}
    \hat e_{1}\\
    \hat e_{2}\\
    \vdots\\
    \hat e_{n}
\end{bmatrix} = 0 \Leftrightarrow \begin{bmatrix}
    \sum_{i=1}^{n}\hat e_{i}\\
    \vdots\\
\end{bmatrix} = 0
\end{align*}
\end{proof}

\begin{definition}[Linear estimator]
An estimator is linear in random variables if it can be written as linear transformation of a random vector. In words, it must be equal to a constant matrix multiplied by a random vector: $$\tilde{\beta} = \tilde{C}Y$$
\end{definition}

\begin{definition}[Projection matrix]
The matrix $P = X\tilde{C}$ is a projection matrix. Its properties are that:\begin{itemize}
\item $PX = X$
\item $P=P'$
\item $PP = P$
\item $\operatorname{tr}(P) = k$
\end{itemize} 
\end{definition}

\begin{remark}
The projection matrix $P$ can be used to recover the fitted values with $$PY = \hat Y$$
\end{remark}
\begin{proof}
We have $PY = X(X'X)^{-1}X'Y = X\hat\beta = \hat Y$
\end{proof}

\begin{definition}[Orthogonal projection matrix]
Let $M = I_n - P$, $M$ is called the orthogonal projection matrix since $MX = 0$. Other properties of $M$ include:\begin{itemize}
\item $MP = 0$
\item $\operatorname{tr}(M) = n-k$
\item $MY = Y - PY = Y - \hat Y = \hat e $
\item $\hat e = MY = M(X\beta + e) = Me$
\end{itemize}
\end{definition}

\begin{definition}[Estimator of the error variance]
The moment estimator of the variance $\sigma^2$ would be: $$\tilde{\sigma}^2 = \frac{1}{n}\sum_{i=1}^n e_i^2 $$ However, $e_i$ is never observed and cannot be used. Let's substitute for $\hat{e_i}$ after OLS estimation. We get the feasible variance estimator:$$\hat{\sigma}^2 = \frac{1}{n} \sum_{i=1}^n \hat e_i^2 $$ Alternatively, we can write $\tilde{\sigma}^2 = n^{-1}e'e$ and $\hat{\sigma}^2 = n^{-1}\hat{e}'\hat{e} = n^{-1}Y'MMY = n^{-1}e'MMe = n^{-1}e'Me$. A nice property of this is that:\begin{align*}
\tilde{\sigma}^2 - \hat{\sigma}^2 = n^{-1}e'e - n^{-1}e'Me & = n^{-1}e'(I_n - M)e \\
& = n^{-1}e'Pe \\
& \geq 0
\end{align*}which means that $\tilde{\sigma}^2 \geq \hat{\sigma}^2$.
\end{definition}

\begin{definition}[$R^2$ and analysis-of-variance]
We can measure the variance of the model with a variable called $R^2$. Write $$Y = PY + MY = \hat{Y} + \hat{e}$$ It follows that $$Y'Y = \hat{Y}'\hat{Y} + 2\hat{Y}'\hat{e} + \hat{e}'\hat{e} = \hat{Y}'\hat{Y} + \hat{e}'\hat{e} $$
And hence $Y - \bar{Y} = \hat{Y} - \bar{Y} + \hat{e} \Rightarrow (Y - \bar{Y})'(Y - \bar{Y}) = (\hat{Y} - \bar{Y})'(\hat{Y} - \bar{Y}) + 2(\hat{Y} - \bar{Y})'\hat{e} + \hat{e}'\hat{e}$ which gives $$\V{Y} = \V{\hat{Y}} + \V{\hat{e}}$$ Finally, we define as $R^2$ the proportion of variation of $Y$ explained by a variation in $\hat{Y}$: $$R^2 = \frac{\V{\hat{Y}}}{\V{Y}} = 1 - \frac{\V{\hat{e}}}{\V{Y}} $$
\end{definition}

We have already seen that, in order to get a solution for our OLS estimator we need the assumption of non-singularity of $X'X$. In the same spirit, we will need other assumptions in order to draw out the properties of $\hat\beta$ whether in finite or infinite samples. The assumptions that are going to be described here represent the minimal assumptions that one can make ; we'll see what they imply and how to relax them in the following sections.

\begin{definition}[Classical assumptions]
The following assumptions on our model are called the classical assumptions:
\begin{itemize}
\item[\textbf{A1}]\textbf{Linearity and correct specification:} the model must be correctly specified as linear in parameters (no $\beta^2$). In matrix form, the model must be represented by $$Y = X\beta + e$$
\item[\textbf{A2}]\textbf{No randomness in $X$:} the data in $X$ is not random (issued by a random variable). It would be the exact same if we took another sample of the population. Mathematically, $$\E{e_i^2|X} = \E{e_i^2}$$
\item[\textbf{A3}]\textbf{Non-singularity of $X'X$:} for it to be non-singular, it must be that $n>k$ (there are more observations than explanatory variables $\Rightarrow$ non over-idenfication) and $\operatorname{rank}(X) = k$ (no multicollinearity in $X$).
\item[\textbf{A4}]\textbf{The errors are spherical:} in particular, $\E{e_i} = \E{e} = 0$ and $\V{e} = \Omega = \sigma^2I_n$. This property also means that there is no heteroskedasticity nor autocorrelation in the data.
\end{itemize}
\end{definition}


\section{Finite sample properties of OLS estimation}
Thanks to these four assumptions, we will be able to discuss more in depth the properties of our OLS estimator, first in finite samples.

\begin{remark}
Under the assumptions A1-A4, the OLS estimator $\hat\beta$ is unbiased.
\end{remark}
\begin{proof}
We already know that $\hat\beta = (X'X)^{-1}X'Y$. Therefore, \begin{align*}
\E{\hat\beta} = \E{(X'X)^{-1}X'Y} & = \E{(X'X)^{-1}X'(X\beta + e)}\\
& = \E{(X'X)^{-1}X'X\beta + (X'X)^{-1}X'e}\\
& = \E{\beta + (X'X)^{-1}X'e}\\
& = \beta + (X'X)^{-1}X'\E{e}\\
& = \beta
\end{align*}
\end{proof}

\begin{remark}
A linear estimator is unbiased if and only if its associated transformation matrix $\tilde{C}$ is such that $$\tilde{C}X = I_k$$
\end{remark}
\begin{proof}
In the case of linear regression, we'd have \begin{align*}
\tilde{\beta} = \tilde{C}(X\beta + e) = \tilde{C}X\beta + \tilde{C}e \Rightarrow \E{\tilde{\beta}} & = \tilde{C}X\beta + \tilde{C}\E{e} \\
& = \tilde{C}X\beta = \beta \text{ if } \tilde{C}X = I_k
\end{align*}This is verified in the OLS since $\tilde{C}X = (X'X)^{-1}X'X = I_k$
\end{proof}

\begin{remark}
The variance of a homoskedastic, non-autocorrelated linear estimator is given by $\V{\tilde{\beta}} =  \sigma^2(\tilde{C}\tilde{C}')$. In the particular case of the OLS estimator, we have $\V{\hat{\beta}_{OLS}} = \sigma^2(X'X)^{-1}$
\end{remark}
\begin{proof}
The proof is trivial and follows the properties of the variance.
\begin{align*}
\V{\tilde{\beta}} = \V{\tilde{C}X\beta + \tilde{C}e} = \V{\tilde{C}e} & = \tilde{C}\V{e}\tilde{C}'\\
& = \tilde{C}\Omega\tilde{C}'\\
& = \tilde{C}\sigma^2 I_n\tilde{C}'\\
& = \sigma^2(\tilde{C}\tilde{C}')
\end{align*}
In the case of $\hat\beta_{OLS}$, we have $\sigma^2(\tilde{C}\tilde{C}') = \sigma^2((X'X)^{-1}X'((X'X)^{-1}X')') = \sigma^2(X'X)^{-1}$
\end{proof}

\begin{remark}
The OLS estimator $\hat{\beta}_{\text{OLS}}$ is BLUE: the Best Linear Unbiased Estimator. This property means that, among linear unbiased estimators, the OLS estimator is the most efficient one.
\end{remark}
\begin{proof}

\end{proof}

\begin{bclogo}[couleur=blue!10, arrondi=0.1, logo=,ombre=false]{ Practical considerations on the data} 
\begin{small}
Assume the model is $$y_i = a + bx_i + e_i$$
Then, we can show that \begin{itemize}
\item $\V{\hat b} = $
\item $\V{\hat a} = $
\item $\cov{\hat b, \hat a} = $
\end{itemize}
Analyzing the data we can find some interesting properties for our model.

For example, if $\sigma^2$ is small, all three variances and covariance will be small as well. A lower $\sigma$ implies a more efficient model.

Now, if $n$ is big, the effect is the same, since all variances will be smaller, our model will be more accurate.

Again, the implications are the same with greater values of $x_i - \bar x$.

Finally, we can see that the covariance between the two estimators indicate how their errors are related. If the covariance is high and positive, then a mistake in the estimation of $\hat b$ will lead to the same mistakes in $\hat a$.
\end{small}
\end{bclogo}

Now we have seen that $\V{\hat{\beta}}$ is given by $\sigma^2(X'X)^{-1}$. However, in actual situations, the true value $\sigma$ is unknown and we would need to estimate it.

\begin{remark}[Properties of the variance estimator]
Let $\hat\sigma^2$ be the sample moment estimator. This estimator is biased and: $$\E{\hat{\sigma}^2} = \sigma^2\left(\frac{n-k}{n}\right)$$
\end{remark}
\begin{proof}
\begin{align*}
\E{\hat\sigma^2} = \E{n^{-1}e'Me} = n^{-1}\E{\operatorname{tr}(Mee')} = n^{-1}\operatorname{tr}(M\E{ee'}) & = n^{-1}\operatorname{tr}(M\Omega) \\ & = n^{-1}\sigma^2(n-k)
\end{align*}
\end{proof}

\begin{definition}[Adjusted sample variance]
We define $s^2$ to be the adjusted sample estimator of the variance, in short the adjusted sample variance, such that: $$s^2 = \frac{\hat e'\hat e}{n-k}=\frac{(Y - X\hat{\beta})'(Y - X\hat{\beta})}{n-k}$$ This implies that this time we have: $\E{s^2} = \sigma^2$. 
Hence, we can use this estimator to estimate the variance of our OLS estimator $\hat\beta$: $$\widehat{\V{\hat{\beta}}} = s^2(X'X)^{-1}$$ Each parameter $\hat{\beta}_k$'s variance would be the $(k,k)$th element of the matrix.
\end{definition}

Again, we find ourselves with more information about $\hat{\beta}$, namely its mean and variance, but not enough information to get the whole distribution of $\hat{\beta}$. We know that $\hat{\beta} = \beta + (X'X)^{-1}X'e$ where the distribution $e$ is the only unknown. We will need a new assumption.
\begin{definition}[Normality of the error term]
Assuming all classical assumptions hold. We add the assumption (\textbf{A5}) that the error term $e_i$ follows a normal distribution of mean $0$ and variance $\sigma^2I_n$.

Following this assumption, we now know that $Y\sim N(X'\beta, \sigma^2 I_n)$ and $\hat{\beta}\sim N(\beta,\sigma^2(X'X)^{-1})$.
\end{definition}

\begin{remark}
Let $V_j$ denote the $(j,j)$th element of the matrix $(X'X)^{-1}$. Then, $\hat{\beta}_j\sim N(\beta_j, \sigma^2V_j)$ where $\sigma^2$ can be estimated with $s^2$.

Therefore, $$\frac{\hat{\beta}_j -\beta_j}{\sqrt{\sigma^2V_j}}\sim N(0,1)$$ while, $$\frac{\hat{\beta}_j -\beta_j}{\sqrt{s^2V_j}}\sim t_{n-k}$$
\end{remark}

This last fact can be used in interval estimation as it implies that: $$\operatorname{Pr}(\hat{\beta}_j - t_{\alpha /2}\frac{S}{\sqrt{V_j}} $$

\begin{remark}[Moments of the residuals]
Let the residuals of the regression be $\hat{e} = Me$ as we've seen before. We have that:\begin{itemize}
\item $\E{\hat{e}} = 0$
\item $\V{\hat{e}} = M\sigma^2$
\end{itemize}
\end{remark}
\begin{proof}
We have that: $\E{\hat{e}} = \E{Me} = M\E{e} = 0$. And $\V{\hat{e}} = \V{Me} = M\V{e}M' = M\Omega M = M\sigma^2I_nM = \sigma^2MM = \sigma^2$
\end{proof}

\

\section{Asymptotic properties of OLS estimation}

\begin{remark}[Consistency of $\hat{\beta}_{\text{OLS}}$]
Let $Q_n = \frac{X'X}{n}$, which is a non-singular, positive definite matrix (from A3). Moreover, let its limit $Q = \lim_{n\to\infty} Q_n$ exist.

Consider that $\E{\hat{\beta}}=\beta$ and $\V{\hat{\beta}} = \frac{\sigma^2}{n}Q_n^{-1}$. Then, \begin{itemize}
\item $\lim_{n\to\infty}\E{\hat{\beta}} = \beta$
\item $\lim_{n\to\infty}\V{\hat{\beta}} = \lim_{n\to\infty} \frac{\sigma^2}{n}Q_n^{-1} = 0$
\end{itemize}
and therefore, $\hat{\beta}\msconv\beta$.
\end{remark}

\begin{remark}[Root-n-consistency of $\hat{\beta}_{\text{OLS}}$]

\end{remark}